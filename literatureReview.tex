\chapter{Literature Review}
The main purpose of this chapter is to present the needed theoretical and technical background to understand how the plugin works. Therefore, a short but sufficient definition of various concepts and tools will be given. In terms of theoretical context, this chapter will discuss topics like  Event-B, Maude and the rewriting logic approach to Event-B. Furthermore, this chapter will also provide some insight into plugin development in Rodin. To guide the reader through the different sections of this chapter and facilitate its reading, the following section dependency is given:

%insert image of section dependencies


\section{Event-B}

\section{Probabilistic Event-B}

\section{Maude}
%explain the generality of Maude
Maude is a high performance declarative language, that allows the specification of programs or systems, and their formal verification \cite{MaudeManual, Lecture1, PeterMaude}.
%explain how Maude program are written in functional module (sorts, subsorts, ops, etc, equations and equation evaluation)
Maude programs are represented as \textit{functional modules} declared with syntax:
\begin{lstlisting}
fmod MODULENAME is
    BODY
endfm
\end{lstlisting}
where \textit{MODULENAME} is the name of the functional module, and \textit{BODY} is a set of declarations that specify the program. The body of the module contains \textit{sorts} (written in Maude as \texttt{sorts}), where each sort correspond to an specific data type of the program. It also contains a set of function symbols or function declarations called \textit{operators} (abbreviated as \texttt{op} in Maude), that specify the constructors of the different sorts, along with the syntax of the program functions. Finally, a set of \textit{equations} (abbreviated as \texttt{eq} in Maude) is used to define the behavior of the functions. These equations use \textit{variables} (abbreviated as \texttt{var} in Maude) to describe how each function works. 

%provide an example of a maude program functional module
To illustrate how a Maude program is contructed, the following code corresponds to a program that defines the natural numbers and addition and multiplication operations: 
\begin{lstlisting}
fmod NAT-ADD is
  sort Nat .

  op 0 : -> Nat [ctor] .
  op s : Nat -> Nat [ctor] .
  op _+_ : Nat Nat -> Nat .
  op _*_ : Nat Nat -> Nat .

  vars N M : Nat .
    
  *** Recursive Definition for addition
  eq N + 0 = N .
  eq N + s(M) = s(N + M) .
endfm
\end{lstlisting}
The sort \texttt{Nat} is a data type that represents the natural numbers. This sort has two constructors (represented in the code with the key word \texttt{ctor}): 0 which is a constant and the operator \texttt{s}, which takes one argument of type \texttt{Nat} and represents the successor function in the natural numbers. With these two operators, it is possible to define arithmetic functions in the natural numbers, like addition or multiplication. In this module, both functions are defined inductively using equations. Using this module, it is possible to compute the value for addition or multiplication for two natural numbers using the command \texttt{red}. For example, if the command \texttt{red s(s(s(0))) + s(s(0))} is used, that represents the operation 3 + 2, the answer 5 is obtained represented as \texttt{s(s(s(s(s(0)))))}:
\begin{lstlisting}
*********** equation
eq N + s(M) = s(N + M) .
N --> s(s(s(0)))
M --> s(0)
s(s(s(0))) + s(s(0))
--->
s(s(s(s(0))) + s(0))
*********** equation
eq N + s(M) = s(N + M) .
N --> s(s(s(0)))
M --> 0
s(s(s(0))) + s(0)
--->
s(s(s(s(0))) + 0)
*********** equation
eq N + 0 = N .
N --> s(s(s(0)))
s(s(s(0))) + 0
--->
s(s(s(0)))
result Nat: s(s(s(s(s(0)))))
\end{lstlisting}
Maude computes using equations from left to right. Therefore in rewriting steps like the first one, the expression \texttt{s(s(s(0))) + s(s(0))} is matched with the left side of the equation \texttt{N + s(M) = s(N + M)} and matching substitution $\{\texttt{N} \mapsto \texttt{s(s(s(0)))},\texttt{M} \mapsto \texttt{s(0)} \}$. The resulting expression \texttt{s(s(s(0))) + s(s(0))}, will be simplified again with the same equation, until it reduces to a simplified expression that can be matched with the equation \texttt{N + 0 = N} (as seen in the last rewriting step).

%explain the semantics behind the functional modules
Semantically speaking, functional modules in Maude are represented as \textit{equational theories} \cite{Lecture1,PeterMaude}, that are represented as a pair $(\Sigma, E)$ where: 
\begin{itemize}
    \item the \textit{signature} $\Sigma$ describes the syntax of the theory, which is the data types and operators symbols (sorts and operators)
    \item $E$ is the set of equations between expressions written in the syntax of $\Sigma$.
\end{itemize}
%explain the functional modules with rewriting rules

%provide an example of a model

%explain theory behind it



\section{Probabilistic Maude (PMaude)}

\section{PVeStA}

\section{A Rewriting Logic Semantics and Statistical Analysis for Probabilistic Event-B}

\section{Rodin and Plugin Development}











\chapter{Introduction}
Nowdays, computer systems are present in people's everyday life. Airplanes, cars, factories, banks or even household appliances integrate these systems in order to function correctly. Considering the influence of computer systems in our society, it is important to construct them in a rigorous manner and verify their correctness, to avoid errors or bugs that can affect negatively people's life. One way of accomplishing this goal is by using formal methods, which use mathematical models for analysis and verification of software or hardware \cite{Woodcock2009}. Formal methods provide a general structure for defining real world systems as abstract models with mathematical rigor, that can be proven to be correct and implemented as specific pieces of software or hardware.   

One of the formal methods used in the software industry is Event-B. It derives from the B method \cite{Abrial1996}, it is semantically based on labeled transition systems (LTS) \cite{LTS} and it is used for specifying discrete distributed systems \cite{Abrial2011}. This method, in conjunction with the Rodin platform \cite{Rodin}, provide a framework for specifying Event-B models and proving mathematical properties over them. Some examples of the uses of Event-B in the design and verification of real world systems include the safety analysis on the CBTC system Octys of the Paris metro lines \cite{Comptier2017} and the formal development of the software for the BepiColombo space mission \cite{Iliasov2010}. 

Even though Event-B proves to be a solid tool for formal specification and verification of systems, throughout its history the Event-B method has been extended with new features that increase its ability to model new systems. For example, Hybrid Event-B incorporates continuous behaviors to the discrete structure of Event-B, which facilitates the construction of cyber physical systems \cite{Banach2015}. Another example is Distributed B: it is an extension that provides the necessary tools for modeling grid systems and verifying their correctness \cite{Grid}.

Another example of  a Event-B extension, is probabilistic Event-B \cite{Morgan2005}, which aims to introduce probabilistic behavior in Event-B. As systems grow in complexity, there is an increasing demand for probabilistic modeling features inside Event-B, where properties like reliability and responsiveness need to be taken into account in the formal verification of these systems \cite{Aouadhi2017}. Therefore, several attempts have been made to move from standard Event-B to probabilistic Event-B. For instance, there is a fully probabilistic extension of Event-B that replaces all non-deterministic choices with probabilistic ones \cite{Aouadhi2017}. Other extensions try to introduce probabilistic choice by using qualitative probabilistic assignments \cite{Hallerstede2007} or quantitative probabilistic choice \cite{Tarasyuk2010}, instead of the non-deterministic assignments used in regular Event-B.

Furthermore, Event-B is not the only specification language where probabilistic extensions have been proposed: Maude \cite{Clavel2007} is a modeling language used to define formal models of distributed systems, based on rewriting logic \cite{Bruni2006}. A probabilistic extension of Maude named PMaude \cite{Agha2006}, permits probabilistic modeling of concurrent systems. Paired with this extension and the query language known as QuaTEx \cite{Agha2006}, statistical model checking tools like PVeStA \cite{AlTurki2011} allow statistical model checking of properties expressed as quantitative temporal logics for PMaude specifications.

Considering the need for specification and verification of probabilistic systems, the available probabilistic extensions of Event-B, and the availability of a framework for probabilistic system specification and verification provided by PMaude and PVeStA,  the authors in \cite{Olarte} present a rewriting logic semantics for a probabilistic extension of Event-B. The previously mentioned paper, provides an automated process for translating an Event-B specification into a probabilistic rewrite theory $\mathscr{R}_\mathscr{M}$, where Monte Carlo simulations can be run using the PVeStA tool, to verify properties over the model written as QuaTEx queries. The translation process is also implemented as a tool that takes an Event-B model and turns it into a PMaude model that is executable using PVeStA, and it is available in \cite{tool.website}.

Despite having a theoretical foundation and an implemented tool for the translation of probabilistic Event-B models to probabilistic rewrite theories, it is also necessary to implement it for the existing Event-B tools like Rodin. Projects like the EBPR project \cite{EBRP} aim to enhance the existing Rodin toolset by allowing to import and use externally defined domain theories as Rodin plugins. Hence, in order to contribute to the Rodin and Event-B capabilities, it is necessary to render extensions, such as the one mentioned previously in \cite{Olarte}, into practical Rodin plugins.